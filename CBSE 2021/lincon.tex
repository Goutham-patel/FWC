\documentclass{article}
\usepackage{siunitx}
\usepackage{setspace}
\usepackage{gensymb}          
\usepackage{xcolor}
\usepackage{caption}
%\usepackage{subcaption}
%\doublespacing               
\singlespacing   
\usepackage[none]{hyphenat}   
\usepackage{amssymb} 
%\usepackage{relsize} 
\usepackage[cmex10]{amsmath}  
\usepackage{mathtools}      
\usepackage{amsmath}   
\usepackage{commath}  
%\usepackage{amsthm}    
%\interdisplaylinepenalty=2500 
%\savesymbol{iint}   
%\usepackage{txfonts}
%\restoresymbol{TXF}{iint}  
%\usepackage{wasysym}    
\usepackage{amsthm}   
\usepackage{mathrsfs}
\usepackage{txfonts}
\let\vec\mathbf{}
%\usepackage{stfloats}
\usepackage{float}
\usepackage{cite}
\usepackage{cases}
\usepackage{subfig}
%\usepackage{xtab}
\usepackage{longtable}
\usepackage{multirow}
%\usepackage{algorithm}
\usepackage{amssymb}
%\usepackage{algpseudocode}
\usepackage{enumitem}
\usepackage{mathtools}
%\usepackage{eenrc}
%\usepackage[framemethod=tikz]{mdframed}
\usepackage{listings}
\usepackage{listings}         
\usepackage[latin1]{inputenc}   
%% \usepackage{color}        
%% \usepackage{lscape}       
\usepackage{titling}                 
%\usepackage{fulbigskip}   
\usepackage{tikz}      
\usepackage{graphicx}
\graphicspath{{/Internal storage/Download/FWC
}}
\usepackage{atbegshi}
%http://ctan.org/pkg/atbegshi
\AtBeginDocument{\AtBeginShipoutNext{\AtBeginShipoutDiscard}}
%
\begin{document}
\begin{center}
\title{ Linears and Conics}
\date{}
\maketitle
\section{Linear}     
\end{center}
\begin{enumerate}
    \item If the two lines
    \begin{center}
        $\vec{L}_1 : \vec{x}=5,\frac{y}{3-\alpha}=\frac{z}{-2}$\\
        $\vec{L}_1 : \vec{x}=2,\frac{y}{-1}=\frac{z}{z-\alpha}$\\
        
    \end{center}
    are perpendicular,then the value of $\alpha$ \\
    \begin{enumerate}
        \item $\frac{2}{3}$
        \item $3$
        \item $4$
        \item $\frac{7}{3}$
    \end{enumerate}
    \item Find the shortest distance between the following lines and hence write
whether the lines are intersecting or not.\\
\begin{center}
    $\frac{x-1}{2}$ = $\frac{y+1}{3}$ = z , $\frac{x+1}{5}$=$\frac{y-2}{1}$,$z=2$
\end{center}
\begin{center}
    $\textbf{OR}$
\end{center}
\ Find the equation of the plane through the line of intersection of the planes \\
    $ \overrightarrow{r}$ .($\hat{i}+3\hat{j}$) + $6$ = $0$  and  $ \overrightarrow{r}$ .($3\hat{i} - \hat{j} - 4\hat{k}) = 0$, which is at a unit distance from the origin.\\
    \item If segment of the line intercepted between the co-ordinate-axes is bisected
at the point $M(2, 3)$, then the equation of this line is
 \begin{enumerate}
     \item $2x + 3y = 13$
     \item $ x + y = 5 $
     \item $ 2x + y = 7$
     \item $3x + 2y = 12 $
 \end{enumerate} 
 
\item The equation of a line through $(2, – 4)$ and parallel to x-axis is $\underline{\hspace{2cm}}$.\\
\item Find the equation of the median through vertex $A$ of the triangle $ABC$, having vertices $A(2, 5), B(– 4, 9)$ and $C(– 2, – 1)$. \\
\item Solve the system of linear equations, using matrix method : 
\begin{center}
    \item $7x + 2y = 11$
    \item  $4x - y = 2$

\end{center}
    
\end{enumerate}
\begin{center}
    \section{Conic}
    \begin{enumerate}
        \item The point at which the normal to the curve  $y = x+\frac{1}{x}$, $x>0$  is perpendicular to the line $3x – 4y – 7 = 0 is:$\\

         a)($2,\frac{5}{2}$) $\hspace{3cm}$   b)($\pm2$,$\frac{5}{2}$)  \\

         c) ($-\frac{1}{2},\frac{5}{2}$) $\hspace{2.8cm}$   d) ($\frac{1}{2},\frac{5}{2}$)\\
         \item The points on the curve $\frac{x^2}{9} +\frac{y^2}{16} = 1$ at which the tangents are parallel to $y$-axis are:\\
           a)($0$,$\pm4$) $\hspace{3cm}$   b)($\pm$4,0)  \\

         c) ($\pm3$,$0$) $\hspace{2.8cm}$   d) ($0$,$\pm3$)\\
         \item For which value of m is the line $y = mx + 1$ a tangent to the curve $y^2 = 4x ?$\\
         a)$\frac{1}{2}$ $\hspace{4cm}$  b) $1$

         c)$2$ $\hspace{4cm}$ d)$3$
    \end{enumerate}
   
       
   

\end{center}

\end{document}
