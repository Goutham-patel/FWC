\documentclass{article}
\usepackage{siunitx}
\usepackage{setspace}
\usepackage{gensymb}          
\usepackage{xcolor}
\usepackage{caption}
%\usepackage{subcaption}
%\doublespacing               
\singlespacing   
\usepackage[none]{hyphenat}   
\usepackage{amssymb} 
%\usepackage{relsize} 
\usepackage[cmex10]{amsmath}  
\usepackage{mathtools}      
\usepackage{amsmath}   
\usepackage{commath}  
%\usepackage{amsthm}    
%\interdisplaylinepenalty=2500 
%\savesymbol{iint}   
%\usepackage{txfonts}
%\restoresymbol{TXF}{iint}  
%\usepackage{wasysym}    
\usepackage{amsthm}   
\usepackage{mathrsfs}
\usepackage{txfonts}
\let\vec\mathbf{}
%\usepackage{stfloats}
\usepackage{float}
\usepackage{cite}
\usepackage{cases}
\usepackage{subfig}
%\usepackage{xtab}
\usepackage{longtable}
\usepackage{multirow}
%\usepackage{algorithm}
\usepackage{amssymb}
%\usepackage{algpseudocode}
\usepackage{enumitem}
\usepackage{mathtools}
%\usepackage{eenrc}
%\usepackage[framemethod=tikz]{mdframed}
\usepackage{listings}
\usepackage{listings}         
\usepackage[latin1]{inputenc}   
%% \usepackage{color}        
%% \usepackage{lscape}       
\usepackage{titling}                 
%\usepackage{fulbigskip}   
\usepackage{tikz}      
\usepackage{graphicx}
\graphicspath{{/Internal storage/Download/FWC
}}
\usepackage{atbegshi}
%http://ctan.org/pkg/atbegshi
\AtBeginDocument{\AtBeginShipoutNext{\AtBeginShipoutDiscard}}
\newcommand{\mydet}[1]{\ensuremath{\begin{vmatrix}#1\end{vmatrix}}}
\providecommand{\brak}[1]{\ensuremath{\left(#1\right)}}
\providecommand{\norm}[1]{\left\lVert#1\right\rVert}
\newcommand{\solution}{\noindent \textbf{Solution: }}
\newcommand{\myvec}[1]{\ensuremath{\begin{pmatrix}#1\end{pmatrix}}}
\let\vec\mathbf
\begin{document}
\begin{center}
\title{ Linears and Conics}
\date{}
\maketitle
\section{Linear}     
\end{center}
\begin{enumerate}
    \item If the two lines
    \begin{align}
          L_1 : x=5,\frac{y}{3-\alpha}=\frac{z}{-2}\\
         L_1 : x=2,\frac{y}{-1}=\frac{z}{z-\alpha} 
       \end{align}
  are perpendicular,then the value of $\alpha$ \\
        \begin{enumerate}
        \item $\frac{2}{3}$
        \item $3$
        \item $4$
        \item $\frac{7}{3}$
    \end{enumerate}

    \item Find the shortest distance between the following lines and hence write
whether the lines are intersecting or not.
\begin{align}
    \frac{x-1}{2} &= \frac{y+1}{3} = z \\
    \frac{x+1}{5} &=\frac{y-2}{1},z=2
\end{align}

\item  Find the equation of the plane through the line of intersection of the planes \\
\begin{align}
     \vec{r} .\brak{i+3j} + 6 &= 0 \\  \vec{r} .\brak{3i - j - 4k} &= 0
\end{align}
which is at a unit distance from the origin.\\
    \item If segment of the line intercepted between the co-ordinate-axes is bisected
at the point $M\brak{2, 3}$, then the equation of this line is
 \begin{align}
       2x + 3y &= 13\\
       x + y &= 5 \\
       2x + y &= 7\\
       3x + 2y &= 12
\end{align}
\item The equation of a line through $(2,-4)$ and parallel to x-axis is $\underline{\hspace{2cm}}$.
\item Find the equation of the median through vertex $A$ of the triangle $ABC$, having vertices $A\brak{2,5}$, $B\brak{-4,9}$ and $C\brak{-2, -1}$.
\item Solve the system of linear equations, using matrix method : 
\begin{align}
  7x + 2y &= 11\\
 4x - y &= 2
\end{align}
\end{enumerate}
\begin{center}
    \section{Conic}
    \begin{enumerate}
        
\item The point at which the normal to the curve \\
\begin{align}
    y = x+\frac{1}{x}, x>0 
\end{align}
 is perpendicular to the line\\
 \begin{align}
     3x-4y-7 = 0 
 \end{align}
\begin{enumerate}
    \item $\brak{2,\frac{5}{2}}$   \item $\brak{\pm2,\frac{5}{2}}$  \\

         \item $\brak{-\frac{1}{2},\frac{5}{2}}$    \item $\brak{\frac{1}{2},\frac{5}{2}}$\\
\end{enumerate}
         
         \item The points on the curve\\
         \begin{align}
             \frac{x^2}{9} +\frac{y^2}{16} = 1
         \end{align}
         at which the tangents are parallel to $y$-axis are:
         \begin{enumerate}
             \item $\brak{0,\pm4}$   \item $\brak{\pm4,0}$ 
	     \item  $\brak{\pm3,0}$   \item $\brak{0,\pm3}$
         \end{enumerate}
           
         \item For which value of m is the line\\
         \begin{align}
            y = mx + 1 
         \end{align}a tangent to the curve \\
        \begin{align}
            y^2 = 4x 
        \end{align}
        \begin{enumerate}
            \item  $\frac{1}{2}$  \item $1$

         \item 2  \item 3
        \end{enumerate}
    \end{enumerate}
\end{center}
\end{document}
